% tikzlibrary.code.tex
%
% Copyright 2015 by Giulio Rasi
%
% This file may be distributed and/or modified
%
% 1. under the LaTeX Project Public License and/or
% 2. under the GNU General Public License.
%
% See the files LICENSE_LPPL and LICENSE_GPL for more details.
\ProvidesPackage{wythoff_package}

\usepackage[utf8]{inputenc}
\usepackage{amsmath}
\usepackage{amsfonts}
\usepackage{amssymb}

\usepackage{graphicx}
\usepackage{tikz}
\usepackage{tikz}
\usepackage[nomessages]{fp}
\usetikzlibrary{arrows,decorations.pathmorphing,decorations.markings,backgrounds,positioning,fit,petri}
\tikzset{fontscale/.style = {font=0.01}
    }

%%Wythoff diagrams
\newcommand{\wyradius}{.08cm}
\newcommand{\wystep}{1 cm}
\newcommand{\wydot}[2]{\fill (#1,#2) circle (\wyradius);}
\newcommand{\wycircle}[2]{
											\fill (#1,#2) circle (0.14);
											\fill[white] (#1,#2) circle (0.11);
											\fill (#1,#2) circle (0.075);
											}
\newcommand{\segments}[1]{
												\pgfmathsetmacro{\start}{#1 -0.25}	
												\pgfmathsetmacro{\fine}{#1 +0.25}	
												\draw[dashed] (\start , 0) -- (\fine,0);

}
%%%% An %%%%%
\newcommand{\wyAn}[1]{
											\foreach \x in {1,2,...,#1}
											{
											\wydot{\x}{0}
											
											}
											\draw (1,0) -- (#1, 0);
										}
										
										
%%%% An with dots %%%%%
\newcommand{\wyAndots}{
											\foreach \x in {1,2,3}
											{
											\wydot{\x}{0}
											}
											\draw (1,0) -- (3.35, 0);
											\segments{4}
											\draw (4.65,0) -- (5,0);
											\wydot{5}{0}
											
										}										
										
%%%% An with dots  WITH LABELES%%%%%
\newcommand{\wyAndotslabs}{
											\foreach \x in {1,2,3}
											{
											\wydot{\x}{0}
											\draw (\x , 0) node[anchor=north] {\tiny{$\x$}};
											}
											\draw (1,0) -- (3.35, 0);
											\segments{4}
											\draw (4.65,0) -- (5,0);
											\wydot{5}{0}
											\draw (5 , 0) node[anchor=north] {\tiny{$n$}};
										}										
																				
										
%%%% Bn with dots %%%%%
\newcommand{\wyBndots}{
											\foreach \x in {1,2,3}
											{
											\wydot{\x}{0}
											}
											\draw (1,0) -- (3.35, 0);
											\segments{4}
											\draw (4.65,0) -- (5,0);
											\wydot{5}{0}
											\draw (1.5,0) node[anchor=south] {$4$};
										}												
										
										
%%%% Bn with dots WITH LABELS%%%%%
\newcommand{\wyBndotslabs}{
											\foreach \x in {1,2,3}
											{
											\wydot{\x}{0}
											\draw (\x , 0) node[anchor=north] {\tiny{$\x$}};
											}
											\draw (1,0) -- (3.35, 0);
											\segments{4}
											\draw (4.65,0) -- (5,0);
											\wydot{5}{0}
											\draw (1.5,0) node[anchor=south] {$4$};
											\draw (5 , 0) node[anchor=north] {\tiny{$n$}};
										}												
										
%%%% Bn %%%%%										
\newcommand{\wyBn}[1]{\foreach \x in {1,2,...,#1}
											{
											\wydot{\x}{0}
											}
										\draw (1,0) -- (#1, 0);
										\draw (1.5,0) node[anchor=south] {$4$};
										}



%%%% Dn %%%%%
\newcommand{\wyDn}[1]{
										\pgfmathsetmacro{\count}{#1 - 2}									
										\wydot{0}{.5}
										\wydot{0}{-0.5}
										\wyAn{\count}
										\draw (0,0.5) -- (1,0) -- (0,-0.5);
										}
										
%%%% Dn with dots %%%%%
\newcommand{\wyDndots}{
										\wydot{0}{.5}
										\wydot{0}{-0.5}
																					\foreach \x in {1,2}
											{
											\wydot{\x}{0}
											}
											\draw (1,0) -- (2.35, 0);
											\segments{3}
											\draw (3.65,0) -- (4,0);
											\wydot{4}{0}

										\draw (0,0.5) -- (1,0) -- (0,-0.5);
										}											
							
							
%%%% Dn with dots WITH LABELES %%%%%
\newcommand{\wyDndotslabs}{
										\wydot{0}{.5}
										\wydot{0}{-0.5}
																					\foreach \x in {1,2}
											{
											\wydot{\x}{0}
											}
											\draw (1,0) -- (2.35, 0);
											
											\segments{3}
											\draw (3.65,0) -- (4,0);
											\wydot{4}{0}

										\draw (0,0.5) -- (1,0) -- (0,-0.5);
										
%%%labels
										\draw (1,0) node[anchor=north] {\tiny{$3$}};
										\draw (2,0) node[anchor=north] {\tiny{$4$}};		
										\draw (4,0) node[anchor=north] {\tiny{$n$}};		
										\draw (0,0.5) node[anchor=north] {\tiny{$1$}};		
										\draw (0,-0.5) node[anchor=north] {\tiny{$2$}};												
										
										}											
													
						

%%%% En %%%%%
\newcommand{\wyEn}[1]{
										\pgfmathsetmacro{\count}{#1 - 1}		
										\wyAn{\count}
										\wydot{3}{1}
										\draw (3,0) -- (3,1);
										}

%%%% En WITH LABESs%%%%%
\newcommand{\wyEnlabs}[1]{
										\pgfmathsetmacro{\count}{#1 - 1}		
										\wyAn{\count}
										\wydot{3}{1}
										\draw (3,0) -- (3,1);
										\foreach \x in {2,3,...,#1}
											 {
												\pgfmathsetmacro{\xco}{\x -1}
												\draw (\xco,0) node[anchor=north]{\tiny{$\x$}};
											}
										\draw (3,1) node[anchor=east]{\tiny{$1$}};
										}
						

									
%%%% F4 %%%%%%
\newcommand{\wyF}{\wyAn{4}
								\draw (2.5,0) node[anchor=south] {$4$};	
								}
%%%% F4 WITH LABELS%%%%%%
\newcommand{\wyFlabs}{\wyAn{4}
								\draw (2.5,0) node[anchor=south] {$4$};	
								\foreach \x in {1,2,3,4}
									{
										\draw (\x,0) node[anchor=north]{\tiny{$\x$}};
									}
								}								
							

%%%% Hn %%%%%%
\newcommand{\wyHn}[1]{\wyAn{#1}
								\draw (1.5,0) node[anchor=south] {$5$};	
								}
								
	
%%%% Hn WITH LABELS%%%%%%
\newcommand{\wyHnlabs}[1]{\wyAn{#1}
								\draw (1.5,0) node[anchor=south] {$5$};	
								\foreach \x in {1,2,...,#1}
									{
										\draw (\x,0) node[anchor=north]{\tiny{$\x$}};
									}								
								
								}							
								

										
%%%% Im %%%%%
\newcommand{\wyIm}[1]{
										\wyAn{2}
										\draw (1.5,0) node[anchor=south] {$#1$};
										}
%%%% Im with labels%%%%%
\newcommand{\wyImlabs}[1]{
										\wyAn{2}
										\draw (1.5,0) node[anchor=south] {$#1$};
										\draw(1,0) node[anchor=north]{\tiny{$1$}};
										\draw(2,0) node[anchor=north]{\tiny{$2$}};

										}

%%%% d- Cube %%%%
\newcommand{\wyCube}[1]{\foreach \x in {1,2,...,#1}
												{
												\wycircle{\x}{0}
												}
											}
											
%%% d-cube from B4 with dots%%%
	\newcommand{\wynCubedots}{
										\wyBndots
										\wycircle{1}{0}
											
										}											
				
%%% d-cube free product dots %%%				
\newcommand{\wynCubedotsfree}{
											\foreach \x in {1,2,3}
											{
											\wycircle{\x}{0}
											}
											\segments{4}
											\wycircle{5}{0}	
										}	
%%% d-cube free product dots %%%				
\newcommand{\wynCubedotsfreek}{
											\foreach \x in {1}
											{
											\wycircle{\x}{0}
											}
											\draw(1.75,0) \node[anchor=north]{$k$ dots}
											\segments{1.75}
											\wycircle{2.5}{0}	
										}											
										
										
												
%%%% d-simplex %%%
\newcommand{\wySimpl}[1]{
											\wyAn{#1}
											\wycircle{1}{0}
											}

%%%% n simplex with dots %%%%%
\newcommand{\wynSimpldots}{
										\wyAndots
										\wycircle{1}{0}
											
										}	





\newenvironment{wythoff}{\begin{tikzpicture}[decoration={markings,mark=at position 0.7 with {\arrow{>}}}]}
{\end{tikzpicture}}


\usepackage{xcolor}
\usepackage{listings}
\lstset
{
    language=[LaTeX]TeX,
    breaklines=true,
    basicstyle=\tt\scriptsize,
    keywordstyle=\color{blue},
    identifierstyle=\color{magenta},
}


%%%%% CUSTOM SCRIPTS %%%%

\newcommand{\kcube}{\begin{wythoff}
\wycircle{1}{0}
\segments{1.75}
\draw (1.75,0) node[anchor=north]{$k$};
\wycircle{2.5}{0}
\end{wythoff}}
\newcommand{\ksimpl}{\textbf{INSERT CODE FOR THE k-simplex }}
\newcommand{\kfourcube}{\textbf{INSERT CODE FOR THE 4 k-cube }}

\newcommand{\highAFamily}{INSERT DRAWING OF THE HIGHEST RANKING A-DIAGRAM WITH $P_k$.}
\newcommand{\highBFamily}{INSERT DRAWING OF THE HIGHEST RANKING B-DIAGRAM WITH $P_k$.}
\newcommand{\highDFamily}{INSERT DRAWING OF THE HIGHEST RANKING D-DIAGRAM WITH $P_k$.}
\newcommand{\highEseven}{INSERT DRAWING OF THE HIGHEST RANKING DIAGRAM ON $E_7$ WITH $P_4$.}
\newcommand{\highEnine}{INSERT DRAWING OF THE HIGHEST RANKING DIAGRAM ON $E_9$ WITH $P_k$.}



%%%% DRAW COMPLETE TABLE %%%%


\newcommand{\wycompletetable}{\begin{tabular}{|c|c|}
\hline 
$A_n$ & 

 \begin{wythoff}
\wyAndotslabs draw (1,0) node[anchor=south] {\phantom{$1$}}; 
\end{wythoff}
\\ 
\hline 
$B_n$ & \begin{wythoff} \wyBndotslabs \  \end{wythoff} \\ 
\hline 

$D_n$ & \begin{wythoff} \wyDndotslabs \end{wythoff} \\ 
\hline 
$E_6$ & \begin{wythoff} \wyEnlabs{6} \end{wythoff} \\ 
\hline 
$E_7$ & \begin{wythoff} \wyEnlabs{7}  \end{wythoff} \\ 
\hline 
$E_8$ & \begin{wythoff} \wyEnlabs{8} \end{wythoff} \\ 
\hline 
$F_4$ & \begin{wythoff} \wyFlabs \end{wythoff} \\
 \hline
$H_3$ & \begin{wythoff} \wyHnlabs{3} \end{wythoff}\\
 \hline
$H_4$ &\begin{wythoff} \wyHnlabs{4} \end{wythoff} \\
 \hline
 
$I_2(m)$ &\begin{wythoff} \wyImlabs{m} \end{wythoff} \\
\hline
\end{tabular} }

